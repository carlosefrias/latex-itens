\documentclass{article}

\usepackage{tikz}
\usepackage{animate}
\usetikzlibrary{math}

\begin{document}
    \begin{center}
        \begin{animateinline}[poster=1, controls={play,step,stop}]{5}
            \multiframe{29}{ri=0+0.05}
            {
                \begin{tikzpicture}[
                    scale=2.5, 
                    extended line/.style={shorten >=-#1,shorten <=-#1},
                    extended line/.default=1cm]
                    \path[use as bounding box] (-1,-1) rectangle (5,3);

                    \tikzmath{
                        function calcF(\x){
                            return 1+(2*\x-\x^2)*ln(\x);
                        };
                    }
                    \draw[->,>=latex, thick] (-.5,0) -- (3,0) node[below]{\(x\)};
                    \draw[->,>=latex, thick] (0,-.5) -- (0,1.8) node[left]{\(y\)};
                    
                    \draw[ultra thick, blue, samples=200, domain=0.0001:2.6] 
                        plot (\x, {1+(2*\x-\x^2)*ln(\x)})
                        node[right]{\(f\left(x\right)=1+\left(2x-x^{2}\right)\ln x\)};
                    \draw[thin, dashed] (1,{calcF(1)}) -- (1,0) node[below]{\(1\)};
                    \draw[red,thick,extended line] (1,{calcF(1)}) -- ({2.4-\ri}, {calcF(2.4-\ri)});
                    \draw[thin, dashed] ({2.4-\ri}, {calcF(2.4-\ri)}) -- (2.4-\ri, 0) node[below]{\(1+h\)};
                    \draw[color=blue, fill=white] (0,{calcF(0.0001)}) circle(1pt);
                    \node at (0,0) [below left]{\(O\)};
                \end{tikzpicture}
            }
        \end{animateinline}
    \end{center}
\end{document}
