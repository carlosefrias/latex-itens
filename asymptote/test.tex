\documentclass[12pt]{article}
% Use this form to include EPS (latex) or PDF (pdflatex) files:
%\usepackage{asymptote}
% Use this form with latex or pdflatex to include inline LaTeX code by default:
\usepackage[inline]{asymptote}
% Use this form with latex or pdflatex to create PDF attachments by default:
%\usepackage[attach]{asymptote}
% Enable this line to support the attach option:
%\usepackage[dvips]{attachfile2}
\begin{document}
% Optional subdirectory for latex files (no spaces):
\def\asylatexdir{}
% Optional subdirectory for asy files (no spaces):
\def\asydir{}

\begin{asydef}
    // Global Asymptote definitions can be put here.
    import three;
    usepackage("bm");
    texpreamble("\def\V#1{\bm{#1}}");
    // One can globally override the default toolbar settings here:
    // settings.toolbar=true;
    \end{asydef}
    Here is a venn diagram produced with Asymptote, drawn to width 4cm:
    \def\A{A}
    \def\B{\V{B}}
    %\begin{figure}
    \begin{center}
    \begin{asy}
    size(4cm,0);

    pen colour1=red;
    pen colour2=green;
    pair z0=(0,0);
    pair z1=(-1,0);
    pair z2=(1,0);
    real r=1.5;
    path c1=circle(z1,r);
    path c2=circle(z2,r);
    fill(c1,colour1);
    fill(c2,colour2);
    picture intersection=new picture;
    fill(intersection,c1,colour1+colour2);
    clip(intersection,c2);
    add(intersection);
    draw(c1);
    draw(c2);
    //draw("$\A$",box,z1); // Requires [inline] package option.
    //draw(Label("$\B$","$B$"),box,z2); // Requires [inline] package option.
    draw("$A$",box,z1);
    draw("$\V{B}$",box,z2);

    pair z=(0,-2);
    real m=3;
    margin BigMargin=Margin(0,m*dot(unit(z1-z),unit(z0-z)));
    draw(Label("$A\cap B$",0),conj(z)--z0,Arrow,BigMargin);
    draw(Label("$A\cup B$",0),z--z0,Arrow,BigMargin);
    draw(z--z1,Arrow,Margin(0,m));
    draw(z--z2,Arrow,Margin(0,m));
    shipout(bbox(0.25cm));
\end{asy}
%\caption{Venn diagram}\label{venn}
\end{center}
%\end{figure}

% \verb+attachfile2+ \LaTeX\ package.
% \begin{center}
% \begin{asy}[height=4cm,inline=true,attach=false,viewportwidth=\linewidth]
% currentprojection=orthographic(5,4,2);
% draw(unitcube,blue);
% label("$V-E+F=2$",(0,1,0.5),3Y,blue+fontsize(17pt));
% \end{asy}
% \end{center}
% One can also scale the figure to the full line width:
% \begin{center}
% \begin{asy}[width=\the\linewidth,inline=true]
% pair z0=(0,0);
% pair z1=(2,0);
% pair z2=(5,0);
% pair zf=z1+0.75*(z2-z1);
% draw(z1--z2);
% dot(z1,red+0.15cm);
% dot(z2,darkgreen+0.3cm);
% label("$m$",z1,1.2N,red);
% label("$M$",z2,1.5N,darkgreen);
% label("$\hat{\ }$",zf,0.2*S,fontsize(24pt)+blue);
% pair s=-0.2*I;
% draw("$x$",z0+s--z1+s,N,red,Arrows,Bars,PenMargins);
% s=-0.5*I;
% draw("$\bar{x}$",z0+s--zf+s,blue,Arrows,Bars,PenMargins);
% s=-0.95*I;
% draw("$X$",z0+s--z2+s,darkgreen,Arrows,Bars,PenMargins);
% \end{asy}
% \end{center}
\end{document}
