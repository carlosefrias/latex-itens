\documentclass[border=1cm]{standalone}

\usepackage{animate}
\usepackage{tikz}

\begin{document}
    \def\R{3} %radius of the bigger circle
    \def\r{1} %radius of the smaller circle
    \begin{animateinline}[poster=0, controls={play,step,stop}]{10}
        \multiframe{73}{ri=0+5.0}
        {
            \begin{tikzpicture}
                \path[use as bounding box] ({-\R-2*\r-1},{-\R-2*\r-1}) rectangle ({\R+2*\r+1},{\R+2*\r+1});

                \draw[thick, ->, >=latex] (-\R-2*\r-0.5,0) -- (\R+2*\r+0.5,0) node[below]{\(x\)};
                \draw[thick, ->, >=latex] (0,-\R-2*\r-0.5) -- (0,\R+2*\r+0.5) node[left]{\(y\)};
                
                \begin{scope}[rotate around={\ri:(0,0)}]                    
                    \draw[gray, blue] (0,0) circle [radius=\R cm];
                    \begin{scope}[rotate around={\R*\ri:(\R+\r,0)}]
                        \draw[red] (\R+\r,0) circle [radius=\r cm];
                        \draw (\R,0) -- (\R+\r,0);
                        \draw[fill] (\R+\r,0) circle [radius=2pt];
                        \draw[fill, red] (\R,0) circle[radius=2pt];
                    \end{scope}
                \end{scope}

                \draw[color=red, thick, domain=0:\ri, smooth, variable=\t, samples=100] 
                    plot ({(\R+\r)*cos(\t)-\r*cos(((\R+\r)/\r)*\t)},{(\R+\r)*sin(\t)-\r*sin(((\R+\r)/\r)*\t)});
                \node at (0,0) [below left]{\(O\)};
            \end{tikzpicture}
        }
    \end{animateinline}
\end{document}
