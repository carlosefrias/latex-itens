\documentclass{article}

\usepackage{tikz}
\usetikzlibrary{calc}

\begin{document}
    \begin{center}
        \def\angle{30}
        \begin{tikzpicture}
            %desenhando os eixos
            \draw [thick, ->, >=stealth] (0,0) -- (-135:2.5) node[below right]{\(x\)};
            \draw [thick, ->, >=stealth] (0,0) -- (5,0) node[below]{\(y\)};
            \draw [thick, ->, >=stealth] (0,0) -- (0,4) node[left]{\(z\)};

            %definindo alguns pontos
            \coordinate (O) at (0,0);
            \coordinate (C) at (2.5, 0.5);

            %Desenhando o plano que contém a base (pontos definidos a olhometro)
            \draw [fill=gray] (0,-0.5) -- (1, -1.4) -- (5, 1.3) -- (3.8, 2.1) -- cycle;
            
            %definindo pontos "extremos" na base do cone
            \coordinate [rotate around={\angle:(C)}] (C1) at ($(C) + (0:1.8cm and 0.6cm)$);
            \coordinate [rotate around={\angle:(C)}] (C2) at ($(C) + (180:1.8cm and 0.6cm)$);
            \coordinate [rotate around={\angle:(C)}] (V) at ($(C) + (90:3cm)$);
            
            %desenhando o cone
            \draw [fill=white, draw=none] (C1) -- (V) -- (C2);
            \draw [fill=white, ultra thick, dashed, rotate around={\angle:(C)}] (C1) arc (5:175:1.8cm and 0.6cm);
            \draw [fill=white, ultra thick, rotate around={\angle:(C)}] (C2) arc (175:365:1.8cm and 0.6cm);
            \draw [ultra thick] (C1) -- (V) -- (C2);
            
            %desenhando a altura do cone
            \draw [ultra thick, dashed] (C) -- (V);

            %desenhando e marcando pontos e
            \draw [fill=black] (C) circle(2pt) node[above right]{\(C\)};
            \draw (V) node[above left]{\(V\)};
            \draw (O) node[left]{\(O\)};
            \draw (4.5,2) node{\(\alpha\)};
        \end{tikzpicture}
    \end{center}
    
\end{document}
