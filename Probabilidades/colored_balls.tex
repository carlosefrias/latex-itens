
\makeatletter

\pgfdeclareradialshading[tikz@ball]{my ball}{\pgfqpoint{5bp}{10bp}}{%
 color(0bp)=(tikz@ball!10!white);
 color(8bp)=(tikz@ball!65!white);
 color(12bp)=(tikz@ball);
 color(25bp)=(tikz@ball!70!black);
 color(50bp)=(black)}

\pgfdeclareradialshading[tikz@ball]{new ball}{\pgfqpoint{12bp}{12bp}}{%
 color(0cm)=(tikz@ball!10!white);
 color(0.5cm)=(tikz@ball!50!white);
 color(0.75cm)=(tikz@ball);
 color(0.9cm)=(tikz@ball!80!black);
 color(1cm)=(tikz@ball!60!black)} 

% code by percusse:
% https://tex.stackexchange.com/a/74237/13304
\pgfdeclareradialshading{ball shadow}{\pgfpointorigin}%
{color(0cm)=(black);
color(2mm)=(gray!70);
color(3mm)=(gray!30);
color(7mm)=(white)
} 

% to make possible use "myball color=..." 
\tikzoption{my ball color}{\pgfutil@colorlet{tikz@ball}{#1}\def\tikz@shading{my ball}\tikz@addmode{\tikz@mode@shadetrue}}

\tikzoption{new ball color}{\pgfutil@colorlet{tikz@ball}{#1}\def\tikz@shading{new ball}\tikz@addmode{\tikz@mode@shadetrue}}

\tikzoption{ball shadow color}{\pgfutil@colorlet{tikz@ball}{#1}\def\tikz@shading{ball shadow}\tikz@addmode{\tikz@mode@shadetrue}}

\newcommand{\ball}[3][white]{%
\begin{scope}[shift = {(#2,#3)}]
	\node[shading=ball shadow,xscale=2,yscale=0.3,circle,minimum size=7mm] 
	at (0.15,-0.225){};
	% test to decide which shading use: special one for white balls
	% not 100% efficient: maybe it's better to use directly a custom
	% white ball shading, but this way one could always use 
	% "my ball color=green" say
	\ifnum\pdf@strcmp{#1}{white}=\z@%  
	\shade[my ball color=#1] (0.15,0.25) circle (0.5);
	\else
	\shade[new ball color=#1] (0.15,0.25) circle (0.5);
	\fi
\end{scope}
}
\makeatother