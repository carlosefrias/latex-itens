\documentclass{article}
\usepackage[hmargin={2cm},vmargin={2cm}]{geometry}
\usepackage{chessboard}
\usepackage{skak}
\usepackage{multicol}
\usepackage{amsmath}
\usepackage{enumitem}

\newlist{itens}{enumerate}{4}
\setlist[itens]{label*=\textbf{\arabic*.}}

\newlist{emulti}{enumerate}{1}
\setlist[emulti]{label=\textbf{(\Alph*)}}

\begin{document}
    \begin{itens}
        \bigskip
        \item Num jogo de xadrez, cada jogador tem na sua posse dezasseis peças: oito peões iguais, duas torres iguais, dois cavalos iguais, dois bispos iguais, um rei e uma rainha.

        Um dos jogadores tem as peças brancas e o outro tem as peças pretas.

        As peças dispoem-se nas duas primeiras filas mais próximas ao jogador. Na segunda fila ficam os peões e na primeira fila ficam as restantes peças na seguinte forma:
        \begin{itemize}
            \item As torres ficam nos extremos
            \item Os cavalos ficam na segunda e penúltima posições
            \item Os bispos ficam na terceira e ante-penúltima posições
            \item O rei e a rainha ficam no centro, ficando a rainha na casa da sua cor (rainha branca em casa branca/rainha preta em casa preta)
        \end{itemize}

        \begin{center}
            \setchessboard{showmover=false}
            \newgame
            \chessboard
        \end{center}

        Colocando ao acaso as oito peças brancas que não são peões na primeira fila, uma por cada casa, qual é a probabilidade de elas ficarem colocadas na posição correta?


        \begin{multicols}{4}
            \begin{emulti}
                \item \(\dfrac{1}{^8C_2\times^6C_2\times^4A_2}\)
                \item  \(\dfrac{2!}{8!}\)
                \item \(\dfrac{6!}{8!}\)
                \item \(\dfrac{2!}{^8C_2\times^6C_2\times^4A_2}\)
            \end{emulti}
        \end{multicols}

        \bigskip

        \bigskip
    \end{itens}
\end{document}