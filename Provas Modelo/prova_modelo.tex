\documentclass[12pt,a4paper]{exam}
\usepackage[utf8]{inputenc}
\usepackage[T1]{fontenc}
\usepackage[portuguese]{babel}
\usepackage{amsmath,amsfonts,amssymb}
\usepackage{graphicx}
\usepackage{enumerate}
\usepackage{multicol}
\usepackage{tikz}
\usepackage{pgfplots}
\usepackage{amsmath}
\pgfplotsset{compat=1.18}

\title{Proposta de Exame Nacional de Matemática A - 12.º Ano}
\author{Carlos Frias}
\date{\today}

\begin{document}

\maketitle

\textbf{Duração:} 150 minutos \hfill \textbf{Data:} \underline{\hspace{3cm}}

\textbf{Nome:} \underline{\hspace{10cm}} \hfill \textbf{Número:} \underline{\hspace{3cm}}

\textbf{Instruções:}
\begin{itemize}
    \item O exame é constituído por dois grupos: Grupo I e Grupo II.
    \item No Grupo I, para cada questão, indique a única opção correta.
    \item No Grupo II, apresente todos os cálculos e justificações necessárias.
    \item Utilize apenas caneta ou esferográfica de tinta indelével, azul ou preta.
    \item Não é permitido o uso de corretor.
    \item É permitido o uso de calculadora gráfica.
\end{itemize}

\vspace{1cm}

\section*{Grupo I}
\textbf{Questões de Escolha Múltipla}

Para cada uma das questões deste grupo, selecione a única opção correta.

\begin{questions}

\question
Considere uma função \( f\), cuja derivada é definida por \( f'(x) = 3x^2 - 4x + 5 \). Qual é o valor do limite seguinte?

\[
\lim_{x \to 2} \frac{f(x) - f(2)}{x^2 - 4}
\]

\begin{choices}
    \choice \(\frac{5}{4}\)
    \choice \(\frac{7}{4}\)
    \choice \(\frac{9}{4}\)
    \choice \(\frac{11}{4}\)
\end{choices}

\question
Qual das expressões seguintes define a derivada de $f(x) = e^{2x} \cdot \ln(x)$?
\begin{choices}
    \choice $f'(x) = 2e^{2x} \cdot \ln(x) + \frac{e^{2x}}{x}$
    \choice $f'(x) = e^{2x} \cdot \ln(x) + \frac{e^{2x}}{x}$
    \choice $f'(x) = 2e^{2x} \cdot \ln(x) - \frac{e^{2x}}{x}$
    \choice $f'(x) = e^{2x} \cdot \ln(x) - \frac{e^{2x}}{x}$
\end{choices}

\question
No plano complexo da figura estão representadas as imagens geométricas dos números complexos \( z \), \( w \), \( t \), \( u \) e \( p \).

\begin{center}
\begin{tikzpicture}
    \draw[->] (-5,0) -- (5,0) node[right] {Re(z)};
    \draw[->] (0,-1) -- (0,4) node[above] {Im(z)};
    \node at (2,1) [circle,fill,inner sep=1.5pt,label=right:{$z$}] {};
    \node at (-3,3) [circle,fill,inner sep=1.5pt,label=below left:{$w$}] {};
    \node at (0,2) [circle,fill,inner sep=1.5pt,label=right:{$t$}] {};
    \node at (-4,4) [circle,fill,inner sep=1.5pt,label=above left:{$u$}] {};
    \node at (-4,3) [circle,fill,inner sep=1.5pt,label=below left:{$p$}] {};
    \node at (0,0) [label=below left:{$O$}] {};
\end{tikzpicture}
\end{center}

Qual é das imagens geométricas corresponde ao número complexo \( i\cdot z^2 + 1 \)?

\begin{choices}
    \choice \(w\)
    \choice \(u\)
    \choice \(p\)
    \choice \(t\)
\end{choices}

\question
Considere o gráfico da função \( f \) representado abaixo:

\begin{center}
\begin{tikzpicture}
\begin{axis}[
    width=0.6\textwidth,
    height=0.4\textwidth,
    axis lines=middle,
    xlabel={\( x \)},
    ylabel={\( y \)},
    xmin=-3, xmax=3,
    ymin=-2, ymax=2,
    %xtick={-3, -2, -1, 0, 1, 2, 3},
    %ytick={-2, -1, 0, 1, 2},
    samples=100,
    domain=-3:3,
]
\addplot[blue, thick] {x^3 - 3*x};
\end{axis}
\end{tikzpicture}
\end{center}

Qual dos seguintes gráficos representa a segunda derivada \( f'' \) da função \( f \)?

\begin{choices}
    \choice 
    \begin{tikzpicture}
    \begin{axis}[
        width=0.3\textwidth,
        height=0.2\textwidth,
        axis lines=middle,
        xlabel={\( x \)},
        ylabel={\( y \)},
        xmin=-3, xmax=3,
        ymin=-2, ymax=2,
        %xtick={-3, -2, -1, 0, 1, 2, 3},
        %ytick={-2, -1, 0, 1, 2},
        samples=100,
        domain=-3:3,
    ]
    \addplot[red, thick] {6*x};
    \end{axis}
    \end{tikzpicture}

    \choice 
    \begin{tikzpicture}
    \begin{axis}[
        width=0.3\textwidth,
        height=0.2\textwidth,
        axis lines=middle,
        xlabel={\( x \)},
        ylabel={\( y \)},
        xmin=-3, xmax=3,
        ymin=-2, ymax=7,
        %xtick={-3, -2, -1, 0, 1, 2, 3},
        %ytick={-2, -1, 0, 1, 2, 3, 4, 5, 6, 7},
        samples=100,
        domain=-3:3,
    ]
    \addplot[red, thick] {6};
    \end{axis}
    \end{tikzpicture}

    \choice 
    \begin{tikzpicture}
    \begin{axis}[
        width=0.3\textwidth,
        height=0.2\textwidth,
        axis lines=middle,
        xlabel={\( x \)},
        ylabel={\( y \)},
        xmin=-3, xmax=3,
        ymin=-2, ymax=2,
        %xtick={-3, -2, -1, 0, 1, 2, 3},
        %ytick={-2, -1, 0, 1, 2},
        samples=100,
        domain=-3:3,
    ]
    \addplot[red, thick] {0};
    \end{axis}
    \end{tikzpicture}

    \choice 
    \begin{tikzpicture}
    \begin{axis}[
        width=0.3\textwidth,
        height=0.2\textwidth,
        axis lines=middle,
        xlabel={\( x \)},
        ylabel={\( y \)},
        xmin=-3, xmax=3,
        ymin=-2, ymax=2,
        %xtick={-3, -2, -1, 0, 1, 2, 3},
        %$ytick={-2, -1, 0, 1, 2},
        samples=100,
        domain=-3:3,
    ]
    \addplot[red, thick] {-6*x};
    \end{axis}
    \end{tikzpicture}
\end{choices}

\question
Considere o retângulo \([ABCD]\) representado abaixo:

\begin{center}
\begin{tikzpicture}
\coordinate (A) at (0,0);
\coordinate (B) at (4,0);
\coordinate (C) at (4,3);
\coordinate (D) at (0,3);

\draw[thick] (A) -- (B) -- (C) -- (D) -- cycle;
\node at (A) [below left] {A};
\node at (B) [below right] {B};
\node at (C) [above right] {C};
\node at (D) [above left] {D};

\draw[red, ->, thick] (A) -- (B) node[midway, below] {\(\vec{AB}\)};
\draw[red, ->, thick] (A) -- (C) node[midway, above left] {\(\vec{AC}\)};
\end{tikzpicture}
\end{center}

Sabe-se que a área do retângulo é \( 12 \) e o perímetro é \( 14 \). Qual é o valor do produto escalar entre os vetores \(\vec{AB}\) e \(\vec{AC}\)?

\begin{choices}
    \choice 12
    \choice 14
    \choice 16
    \choice 18
\end{choices}

\end{questions}

\section*{Grupo II}
\textbf{Questões de Desenvolvimento}

Nas questões deste grupo, apresente todos os cálculos e justificações necessárias.

\begin{questions}

\question
Considere a função \(f\), de domínio \(\mathbb{R}\), definida por\( f(x) = e^{x} \cdot (x^2 - 3x + 2) \).

\begin{parts}
    \part Determine os zeros da função \( f \).
    \part Estude \( f \) quanto à monotonia, identificando os intervalos onde a função é crescente ou decrescente. Determine também a existência de extremos relativos (máximos ou mínimos relativos), caso existam.
\end{parts}

\question
Considere a função \( g(x) = \frac{1}{x^2 + 1} \).

\begin{parts}
    \part Mostre que a derivada de \( g \) é dada por \( g'(x) = -\frac{2x}{(x^2 + 1)^2} \).
    \part Estude a concavidade da função e identifique os pontos de inflexão, caso existam.
\end{parts}

\question
Resolva a seguinte inequação:

\[
\log_2(x) + \log_2(x - 2) \geq 3
\]

Apresente a solução na forma de intervalo ou união de intervalos.

\question
Numa turma de 12.º ano, há 20 alunos, dos quais 12 são raparigas e 8 são rapazes. O professor vai escolher aleatoriamente 4 alunos para formar uma comissão.

\begin{parts}
    \part Quantas comissões diferentes podem ser formadas?
    \part Qual é a probabilidade de a comissão ser constituída por 2 raparigas e 2 rapazes?
    \part Qual é a probabilidade de a comissão incluir pelo menos uma rapariga?
\end{parts}

\question
Considere a equação trigonométrica:

\[
2\sin^2(x) - 3\sin(x) + 1 = 0
\]

\begin{parts}
    \part Mostre que a equação pode ser escrita na forma \( (2\sin(x) - 1)(\sin(x) - 1) = 0 \).
    \part Resolva a equação no intervalo \( [0, 2\pi] \).
\end{parts}

\question
No espaço tridimensional, considere os pontos \( A(4, 0, 0) \), \( B(0, 2, 0) \) e \( C(0, 0, 3) \).

\begin{center}
\begin{tikzpicture}[scale=1]
    % Draw axes
    \draw[->] (0,0,0) -- (5,0,0) node[anchor=north east]{$x$};
    \draw[->] (0,0,0) -- (0,3,0) node[anchor=north west]{$y$};
    \draw[->] (0,0,0) -- (0,0,4) node[anchor=south]{$z$};

    % Draw points
    \coordinate (A) at (4,0,0);
    \coordinate (B) at (0,2,0);
    \coordinate (C) at (0,0,3);

    % Draw triangle
    \draw[thick] (A) -- (B) -- (C) -- cycle;

    % Label points
    \node[anchor=south] at (A) {$A$};
    \node[anchor=north east] at (B) {$B$};
    \node[anchor=south] at (C) {$C$};
\end{tikzpicture}
\end{center}

\begin{parts}
    \part Mostre que o plano \(ABC \) é definido por \(3x + 6y + 4z = 12 \).
    \part Escreva uma condição que defina a esfera de centro \( O(0, 0, 0) \) e tangente ao plano \( ABC \).
\end{parts}

\end{questions}

\vspace{2cm}

\textbf{Fim do Exame}

\end{document}